\documentclass[UTF8]{ctexart}
\usepackage{geometry}
\geometry{a4paper,margin=1in}
\usepackage{hyperref}
\usepackage{amsmath,amssymb}
\usepackage{enumitem}
\usepackage{graphicx}
\title{MNIST 手写数字分类对比实验指南(MLP/CNN/MinGRU/Mamba/Transformer)}
\author{作者}
\date{\today}
\begin{document}
\maketitle
\begin{abstract}
本文在统一的数据处理、训练与评估流程下,对比五种骨干网络(MLP、CNN、MinGRU、Mamba、Transformer)在 MNIST 及相关数据集上的分类性能,提供快速开始指南、调参建议、结果解读与扩展实验方向,并汇总输出文件与命令。
\end{abstract}
\tableofcontents
\section{项目简介}
\begin{itemize}[leftmargin=*]
\item 对比五种骨干网络在 MNIST 分类任务上的性能:\texttt{mlp}、\texttt{cnn}、\texttt{mingru}、\texttt{mamba}、\texttt{transformer}
\item 统一数据处理、训练与评估流程;输出训练/测试损失与准确率曲线以及指标汇总
\end{itemize}
\section{环境准备}
\begin{itemize}[leftmargin=*]
\item 依赖:\texttt{torch}、\texttt{torchvision}、\texttt{numpy}、\texttt{matplotlib}
\item 安装命令:\verb|pip install -r classify/requirements.txt|
\item 设备:默认 CPU;如需 GPU,可在本地配置 CUDA 并修改代码中的设备
\end{itemize}
\section{快速开始}
\begin{itemize}[leftmargin=*]
\item 运行基础对比(MNIST):\verb|python -m classify.main|
\item 运行多数据集消融:\verb|python -m classify.experiments|
\item 结果位置:\texttt{classify/results/}
\begin{itemize}
\item 单模型曲线:\texttt{\{dataset\}\_\{model\}\_\{variant\}\_curves.png}(如 \texttt{mnist\_cnn\_base\_curves.png})
\item 横向对比(同数据集不同模型):测试准确率 \texttt{\{dataset\}\_cross\_model\_test\_acc.png};测试损失 \texttt{\{dataset\}\_cross\_model\_test\_loss.png}
\item 纵向消融(同模型不同结构/参数):测试准确率 \texttt{\{dataset\}\_\{model\}\_ablation\_test\_acc.png};测试损失 \texttt{\{dataset\}\_\{model\}\_ablation\_test\_loss.png}
\item 指标汇总:\texttt{ablation\_summary.json}
\end{itemize}
\end{itemize}
\section{运行说明}
\begin{itemize}[leftmargin=*]
\item 训练与评估流程在 \texttt{classify/train.py} 中实现,\texttt{classify/main.py} 会顺序运行五种模型并保存结果
\item 单模型运行示例(在交互式环境中调用):\verb|from classify.train import train_one|,\verb|train_one('cnn', epochs=3, batch_size=64, lr=1e-3, device='cpu')|
\end{itemize}
\section{配置与调参}
\subsection{主要参数}
\begin{itemize}[leftmargin=*]
\item \texttt{epochs}:训练轮次(默认 3,建议 5--10 以获得更稳定结果)
\item \texttt{batch\_size}:批次大小(默认 64)
\item \texttt{lr}:学习率(默认 \texttt{1e-3})
\item \texttt{device}:设备(\texttt{cpu} 或 \texttt{cuda})
\end{itemize}
\subsection{修改方式}
\begin{itemize}[leftmargin=*]
\item 批量运行时可编辑 \texttt{classify/train.py} 中 \texttt{run\_all} 的调用参数
\item 单模型运行时直接传参给 \texttt{train\_one}
\end{itemize}
\section{输出与可视化}
每个模型训练结束会生成两类曲线(统一坐标:$\mathrm{loss}\in[0,1]$,$\mathrm{acc}\in[0,1]$):
\begin{itemize}[leftmargin=*]
\item 训练/测试损失随 epoch 变化:左图
\item 训练/测试准确率随 epoch 变化:右图
\end{itemize}
对比图:横向(跨模型)与纵向(同模型不同参数/结构)均以“所有对比项画在一张图”输出,分别提供测试集损失与准确率两张图。
\section{实验结果摘要}
设置:\texttt{3 epoch}, \texttt{batch=64}, Adam \texttt{lr=1e-3}, CPU。
\begin{itemize}[leftmargin=*]
\item \texttt{cnn}:Test Acc 约 \texttt{0.990};Test Loss 约 \texttt{0.032};收敛速度快,稳定性好,训练/测试曲线贴合
\item \texttt{mlp}:Test Acc 约 \texttt{0.973};Test Loss 约 \texttt{0.089};表现稳健,收敛良好,但相对 CNN 略逊
\item \texttt{mingru}:Test Acc 约 \texttt{0.982};Test Loss 约 \texttt{0.065};序列建模对 MNIST 有增益,优于 MLP
\item \texttt{transformer}:Test Acc 约 \texttt{0.964};Test Loss 约 \texttt{0.115};简化 ViT,在更长训练与更大维度下可进一步提升
\item \texttt{mamba}:Test Acc 约 \texttt{0.928};Test Loss 约 \texttt{0.253};简化实现下短训较敏感,适当加深/增大状态维度可改善
\end{itemize}
\section{结果分析与解读}
\begin{itemize}[leftmargin=*]
\item 收敛速度:\texttt{cnn} 与 \texttt{mingru} 收敛快且平滑;\texttt{mlp} 次之;\texttt{transformer} 与 \texttt{mamba} 需更长训练与更大容量
\item 泛化能力:训练/测试曲线间隙小的模型更稳健,\texttt{cnn}、\texttt{mingru} 在当前设置下泛化较好
\item 架构差异:\texttt{cnn} 在图像任务上具备局部建模与平移不变性优势;\texttt{mingru} 将图像按列序列化,门控机制帮助捕捉跨步依赖;\texttt{transformer} 依赖足够的 token 容量与层数,简化配置下性能一般;\texttt{mamba} 的状态空间机制强调长序列效率,图像上需更适配设计与更深层
\end{itemize}
\section{三数据集对比分析(统一坐标)}
\begin{itemize}[leftmargin=*]
\item MNIST:\texttt{cnn} 收敛最快、准确率最高;\texttt{mingru} 次之;\texttt{mlp} 稳健;\texttt{transformer/mamba} 需更大容量及更长训练
\item CIFAR-10:彩色、复杂背景更凸显 \texttt{cnn} 的局部归纳偏置优势;小型 \texttt{transformer} 随容量提升(embed/head/depth)有明显增益;\texttt{mlp} 由于缺少空间结构表现较弱
\item Sequential MNIST(Permuted):长序列依赖更能体现 \texttt{MinGRU/Mamba/Transformer} 差异;\texttt{cnn} 因失去二维先验表现较弱;\texttt{mlp} 亦逊色
\end{itemize}
\section{常见问题与解决}
\begin{itemize}[leftmargin=*]
\item MNIST 下载失败:检查网络或更换下载镜像;确保 \texttt{torchvision} 版本满足要求
\item 训练过慢:减少 \texttt{epochs}/\texttt{batch\_size},或在具备环境的情况下改为 \texttt{device='cuda'}
\item 曲线不收敛:降低学习率至 \texttt{5e-4/1e-4},增加 \texttt{epochs} 至 5--10
\item 精度不达预期:\texttt{cnn} 增加通道宽度与层数,在分类器处增加隐藏层;\texttt{transformer} 增大 \texttt{embed\_dim}、\texttt{num\_heads}、\texttt{depth},添加 \texttt{dropout};\texttt{mamba} 增加 \texttt{state\_size},堆叠多层 cell,延长训练
\end{itemize}
\section{扩展实验建议}
\begin{itemize}[leftmargin=*]
\item 增加训练轮次:将 \texttt{epochs} 调整为 10--20,观察曲线趋稳与最终指标提升
\item 调优策略:余弦退火或阶梯式学习率衰减;在 \texttt{transformer} 中尝试 AdamW
\item 架构扩展:\texttt{cnn} 双卷积块(\verb|1->32->64|)、批归一化、\texttt{Dropout(0.5)};\texttt{mingru} 加大 \texttt{hidden\_size} 至 \texttt{256},或双向扫描再融合;\texttt{mamba} 多层堆叠、提高 \verb|state_size>=128|,尝试可学习位置编码融合;\texttt{transformer} \verb|patch_size=2| 增加 token 数量,\verb|embed_dim=128~256|,\verb|depth=4~6|
\end{itemize}
\section{复现实验与对比}
\begin{itemize}[leftmargin=*]
\item 建议固定随机种子与数据加载参数,以便不同架构可直接横向对比
\item 在相同 \texttt{epochs/batch\_size/lr} 条件下进行测试,并统一评估指标与绘制曲线
\end{itemize}
\section{文件与命令汇总}
\begin{itemize}[leftmargin=*]
\item 曲线文件:\texttt{classify/results/\{dataset\}\_\{model\}\_\{variant\}\_curves.png}
\item 对比图:\texttt{\{dataset\}\_cross\_model\_test\_acc.png}/\texttt{\{dataset\}\_cross\_model\_test\_loss.png}、\texttt{\{dataset\}\_\{model\}\_ablation\_test\_acc.png}/\texttt{\{dataset\}\_\{model\}\_ablation\_test\_loss.png}
\item 指标汇总:\texttt{classify/results/ablation\_summary.json}
\item 安装依赖:\verb|pip install -r classify/requirements.txt|
\item 运行全部:\verb|python -m classify.main|
\end{itemize}
\end{document}
